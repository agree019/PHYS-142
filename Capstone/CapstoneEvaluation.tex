% General Paper Template created by Adam Green
% Last revised 3/2/18

\documentclass[12pt]{article}

%%%%%%%%%%%%%%%%%%%%%%%%%%%%
% Standard Packages
%%%%%%%%%%%%%%%%%%%%%%%%%%%%
\usepackage{amsmath,amsfonts,amsthm,amssymb}
\usepackage[margin = 2cm]{geometry}


%%%%%%%%%%%%%%%%%%%%%%%%%%%%
% Graphical Packages
%%%%%%%%%%%%%%%%%%%%%%%%%%%%
\usepackage{graphicx}
\usepackage{subfig}
\usepackage{float}
\usepackage{tikz}
\usepackage{wrapfig}
% Syntax: \begin{wrapfig}[lineheight]{position}{width}

%\usepackage{tikz-3dplot}
%\usepackage{pgfplots}

\graphicspath{{Figures/}} % set directory for figures

%%%%%%%%%%%%%%%%%%%%%%%%%%%%
% Colors
%%%%%%%%%%%%%%%%%%%%%%%%%%%%
\usepackage{xcolor}

\definecolor{myblue1}{RGB}{76, 142, 185}
\definecolor{myblue2}{RGB}{25, 100, 126}
\definecolor{myblue3}{RGB}{41, 110, 180}

\definecolor{mygreen1}{RGB}{88,165,87}
\definecolor{mygreen2}{RGB}{91,165,98}

\definecolor{myred1}{RGB}{221, 28, 26}
\definecolor{mypurple}{RGB}{122,48,108}


%%%%%%%%%%%%%%%%%%%%%%%%%%%%
% Table and Array Packages
%%%%%%%%%%%%%%%%%%%%%%%%%%%%
\usepackage{tabu}
\usepackage{booktabs}


%%%%%%%%%%%%%%%%%%%%%%%%%%%%
% Citation Packages
%%%%%%%%%%%%%%%%%%%%%%%%%%%%
\usepackage{hyperref}
\hypersetup{allcolors = myblue1,
	allbordercolors = myblue1, 
	filecolor = myblue1, 
	linkbordercolor = myblue1,
	urlbordercolor = white
}

\usepackage{cleveref}
\crefname{equation}{Eqn.}{Eqns.}
\crefname{figure}{Fig.}{Figs.}
\crefname{table}{Tab.}{Tabs.}

%\usepackage{caption}
%	\captionsetup{format = , justification= }

%%%%%%%%%%%%%%%%%%%%%%%%%%%%
% Text Formatting packages
%%%%%%%%%%%%%%%%%%%%%%%%%%%%
\usepackage{multicol}
\usepackage{parskip} 
\usepackage{indentfirst} 
\setlength{\parindent}{0.75cm}
\usepackage{fancyhdr}
%	\pagestyle{fancy}
\usepackage{enumerate}
\usepackage{wrapfig}
\usepackage{lipsum}
\usepackage{xhfill}


%%%%%%%%%%%%%%%%%%%%%%%%%%%%
% Custom Commands
%%%%%%%%%%%%%%%%%%%%%%%%%%%%
% Red marks to get attention
\newcommand{\red}[1]{\textbf{\textcolor{myred1}{#1}}} % Red Text
\newcommand{\redmark}{\textcolor{myred1}{\rule{4mm}{4mm} }} % Red Dash
\newcommand{\redline}{\noindent\xhrulefill{myred1}{3pt}} % Red Rule


% Lowercase Captial Letters
\newcommand{\scap}[1]{\textsc{\MakeLowercase{#1}}} % Makes caps small so it doesnt SHOUT


% Math Commands: first and second order partials, Laplacian
\newcommand{\evaluate}{\Bigr\rvert}
\newcommand{\ppd}[1]{\frac{\partial}{\partial#1}}
\newcommand{\ppsd}[1]{\frac{\partial^2}{\partial #1^2}}
\newcommand{\ppnd}[2]{\frac{\partial #1}{\partial #2}}
\newcommand{\ppsnd}[2]{\frac{\partial^2 #1}{\partial #2^2}}
\newcommand{\lap}{\nabla^2}

% Quantum bra- ket- commands
\newcommand{\bra}[1]{\langle #1 |}
\newcommand{\ket}[1]{| #1 \rangle}
\newcommand{\bracket}[2]{\langle #1 | #2 \rangle}

% Redfine equation and figure references to include "Eqn. ()" and "Fig. _"
%\newcommand{\figref}[1]{Fig.\ \ref{#1}}
%\let\originaleqref=\eqref
%\renewcommand{\eqref}{Eqn.\ \originaleqref}


% Custom Matrix Spacing
% Syntax: \begin{matrix}[scale]
\makeatletter
\renewcommand*\env@matrix[1][\arraystretch]{%
	\edef\arraystretch{#1}%
	\hskip -\arraycolsep
	\let\@ifnextchar\new@ifnextchar
	\array{*\c@MaxMatrixCols c}}
\makeatother


% Email Commands: Taken from Flip
% Syntax: \email{email}
\newcommand{\email}[1]{\href{mailto:#1}{\textcolor{mygreen1}{#1}}}

% Institution Environment: Taken From Flip
\newenvironment{institutions}[1][2em]{\begin{list}{}{\setlength\leftmargin{#1}\setlength\rightmargin{#1}}\item[]}{\end{list}}

% Link to External Webpage
% Syntax: \link{url}
\newcommand{\link}[1]{\href{#1}{\textcolor{mygreen1}{\texttt{#1}}}}


%%%%%%%%%%%%%%%%%%%%%%%%%%%%
% Document Specific Commands
%%%%%%%%%%%%%%%%%%%%%%%%%%%%
\usepackage{setspace}
\doublespacing


\begin{document}
	\begin{center}
		{\LARGE \textbf{Capstone Evaluation: ``Dark Photons from the Center of the Earth''}}\\
		
		{ArXiV: \href{https://arxiv.org/abs/1509.07525v3}{\textcolor{mygreen1}{1509.07525v3}}}
		
		\vspace{0.5cm}
		
		\textbf{Adam Green}\\ 
		
		\texttt{\footnotesize \email{agree019@ucr.edu}}
		
		
		\begin{institutions}[2.25cm]
			\footnotesize
			{\it 
				Department of Physics \& Astronomy, 
				University of  California, Riverside, 
				CA 92521	    
			}    
		\end{institutions}
		\vspace{0.5cm}
	\end{center}

	\vspace{0.5cm}
	
	This paper presents a phenomenological study of a dark sector with a massive spin-1 gauge boson. The authors present the background information on the theory and then present a novel way to test it. 

	
	What the scientific community has been calling ``dark matter'' may actually be a dark sector, dark gauge bosons in addition to dark matter particles. This paper considers a dark sector mediated by a massive spin-1 dark gauge boson, the ``dark photon,'' which kinetically mixes with the Standard Model photon. The dark photon and standard model photon mix according to the mixing parameter $\epsilon$ which is constrained to be on the order of $\mathcal{O}(10^{-5} - 10^{-10})$. When the mass of the dark photon is much less than the mass of dark matter, some interesting effects arise which give rise to new ways to search for dark sectors. In this framework, dark matter will collect at the center of the Earth, annihilate into dark photons which may decay near the surface into detectable leptons. 
	
	As the Earth traverses the Milky way, it is constantly intercepting dark matter along its journey. Occasionally, a dark matter particle will collide with a nucleus within the Earth. If the dark matter particle imparts enough energy into the Earth, it becomes gravitationally captured and falls to the center of the Earth. The capture rate depends on the density of Earth, the relative speed between the earth and dark matter, and the amount of energy imparted into the Earth, called recoil energy. To calculate the capture rate, we integrate these quantities over the volume of earth, all allowed incident velocities, and all allowed recoil energies. This particular capture scenario is interesting because the Feynman diagram describing it is the exact same for direct detection. In this case however, the ``detector'' is the entire Earth.
	
%	We assume that the dark matter halo obeys a Maxwell-Boltzmann velocity distribution in the galactic frame. The speed of the Earth in the galactic frame can be calculated by a series of Galilean transforms. 
	
	At the center of the Earth, captured dark matter will annihilate with itself into dark photons. The rate of annihilation depends on the amount of dark matter that has been captured and the thermally-averaged annihilation cross section. For the special case when the dark photon mass is much lighter than the mass of dark matter, the annihilation rate experiences a \emph{Sommerfeld enhancement}. This enhancement arises when the dark photon mass is light compared to dark matter. Due to dark matter self-interactions, these interactions are now allowed to take place over longer ranges. Now, instead of dark matter annihilations occurring purely due to statistics, dark matter now ``feels'' a slight attraction to itself. These are seen as resonance peaks in the kinetic mixing parameter. The result from these dark matter annihilations are dark photons. This is a classic beam dump experiment, except on a planetary scale.
	
	These dark photons propagate from the center of the Earth where, with some probability, they may interact with detectable charged leptons near the surface. Upon their creation, these dark photons are relativistic because their mass is so small compared to dark matter. They propagate through the Earth with essentially no interactions. We can calculate their decay length from their Lorentz boost and their branching ration into Standard Model particles. This decay length is inversely proportional to the square of the kinetic mixing parameter, and the mass of the dark photons. If the mixing paramter is high, the dark photons are more likely to decay into standard model particles. Similarly, if the mass of the dark photons is high, it costs more energy to move them and they do not travel as far before they decay. To obtain decays near the surface of Earth, the authors determine that the mixing parameter must lie between $10^{-10} - 10^{-5}$ and the dark photon must have a mass between $10 \ \text{MeV} - 100 \ \text{GeV}$. 
	
%	At the surface of the Earth, the signals we expect would come from photon interactions, not dark photon interactions. The model in the paper considers dark photons which mix with standard model photons. However, in the Earth capture scenario, the propagating dark photons are relativistic. Upon their decay into standard model photosns, when they mix with standard model photons

	At the surface of Earth, the dark photon may decay into detectable standard model leptons. To make predictions about their detection signature, the authors characterize the typical size of the accumulated dark matter at the center of Earth. Given that the dark photons are relativistic, when they decay, their daughter particles will have an angular dispersion of $1.3\deg$ from straight down. The trademark detection signature for this process are highly collimated jets of leptons which point back to the center of the earth to within a few degrees, the so-called ``smoking gun'' signal. This signal is particularly useful because there are no other known processes with this signal. Perhaps high energy gamma rays could reproduce this kind of signal, but the chances of that are so low they are negligible. 
	
%	Finally, the authors present preliminary predictions for two classes of detectors, underground and space-based detectors. For underground detectors, they use \scap{icecube} as an example. \scap{icecube} is located at the south pole and uses the high-purity ice as a medium to detect particle events. 
	
	
	
\end{document}